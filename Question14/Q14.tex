\section{Feature selection}
\textbf{Goal}: reduce the number of features used by the learning algorithm on order to: \begin{enumerate}
    \item Avoid overfiiting (eviter l'overfiiting): \textcolor{ae}{some feature might be "irrelevant" for the problem}, \textcolor{red}{fittingthem = overfitting}}
    \item Improve interpretability 
    \item provide faster and more cost-eff model
    \item reduce computing time: ⚠  if fast feature selection
\end{enumerate}

There are two type of feature selection techniques.
\begin{enumerate}
    \item \textbf{Features selection} : Find a small or smallest subset of features that maximize accuracy.
    \item \textbf{Features ranking}: sort the varibles according to ther relevance at predicting output. 
\end{enumerate}

\subsection{Filter techniques}
Filter techniques is a priori selection of the varibles. Independent of learning algo.
\textbf{Method}\begin{enumerate}
    \item Associate a relevance score to cach variable : "Score explained"
    \item Remove low-scoring Features \textcolor{ao}{Treshold neeeded}
\end{enumerate}
There are two type of "\textbf{scoring}"
\begin{enumerate}
    \item \textbf{Univariate scoring}: score each feature independdently. \Example stat test (t-test,..)
    \item \textbf{Multivariate scoring}: score features dependently. Some features are useless alone \textbf{But} with others they perfectly representthe LS. 
\end{enumerate}




  \begin{table}[!h]
    \begin{center}
    \begin{tabular}{| m{8em}| m{25em}|}
    \hline
    \rowcolor{vert.g} \textbf{Advantages}     &  \begin{itemize}
    \item Univariate fitten tech : \textbf{Fast \& scalable}
    \item independent of the learning \algo
\end{itemize}\\ \hline 
    \rowcolor{red.g} \textbf{Drawbacks}       &  \begin{itemize}
    \item Independant of te learning algo: can't be totally efficient.
    \item Univariate: Ignore feature dependecie.
    \item Multivariate: \textbf{slower} than univariate.
\end{itemize}\\ \hline
    \end{tabular}
    \end{center}
    \end{table}



\Example Multivariate scoring: Decision Trees,...
%--------------------------------- Embedded Techniques --------------------------
\subsection{Embedded Techniques}
Some learning algorithms embedded some feature selection: \textbf{Search for an optimal subset} of features is build into the LS.\\
\Example:
\begin{enumerate}
    \item Decision tree node spliting: feature selection techninques 
    \item Weight in svm Model
\end{enumerate}



  \begin{table}[!h]
    \begin{center}
    \begin{tabular}{| m{8em}| m{25em}|}
    \hline
    \rowcolor{vert.g} \textbf{Advantages}     &  \begin{itemize}
    \item computationnaly efficient. 
    \item Well integrated in the learning \algo 
    \item Multivarite 
\end{itemize}\\ \hline 
    \rowcolor{red.g} \textbf{Drawbacks}       &  \begin{itemize}
    \item learning \algo specific
\end{itemize}\\ \hline
    \end{tabular}
    \end{center}
    \end{table}

%--------------------------------- wappen technique --------------------------
\subsection{wappen technique}
Find a subset of features that maximizes the "\textbf{quality of the model"induced by the learning algorithm}


%--------------------------------- Selection Biais --------------------------

% ------------------------------ Selection biais ------------------------------
\subsection{Selection \biais}
\textbf{Protocol}:
\begin{enumerate}
    \item Divide LS into 10 folds
    \item  For i = 1 to 10: 
    \begin{enumerate}
        \item Remove the $i^{th}$ fold from LS $\to LS_i$
        \item select top 20 variables from LS
        \item learn model using the 20 varibles on LS 
        \item Test model on the $i^{th}$ fold 
    \end{enumerate}
\end{enumerate}