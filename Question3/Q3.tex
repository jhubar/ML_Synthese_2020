\section{Model evaluation}
%-------------------------------------------- Resubstitution error -------------------------------
\subsection{Resubstitution error}
Fit the model on a traising set and use the same traising set for model.


  \begin{table}[!h]
    \begin{center}
    \begin{tabular}{| m{8em}| m{15em}|}
    \hline
    \rowcolor{vert.g} \textbf{Advantages}     &  \begin{itemize}
                                                        \item \textbf{Fast} and \textbf{simple}  
                                                 \end{itemize}\\ \hline 
    \rowcolor{red.g} \textbf{Drawbacks}       &  \begin{itemize}
                                                        \item \textbf{High-Biais} due to overtting 
                                                        \item (model may be good on the TS but not on the whole distribution)
                                                 \end{itemize}\\ \hline
    \end{tabular}
    \end{center}
    \end{table}
    
%-------------------------------------------- Test set Method -------------------------------
\subsection{Test set Method}
Randomely divide the dataset into 2 part: (\textbf{LS} 70\%, \textbf{TS} 30\%)

  \begin{table}[!h]
    \begin{center}
    \begin{tabular}{| m{8em}| m{15em}|}
    \hline
    \rowcolor{vert.g} \textbf{Advantages}     &  \begin{itemize}
                                                        \item \textbf{Fast} and \textbf{simple} 
                                                        \item \textbf{Low bias}
                                                 \end{itemize}\\ \hline 
    \rowcolor{red.g} \textbf{Drawbacks}       &  \begin{itemize}
                                                        \item \textbf{Error is not reliable}
                                                        \item (if Dataset is to small)
                                                 \end{itemize}\\ \hline
    \end{tabular}
    \end{center}
    \end{table}
    
        
%-------------------------------------------- K-fold cross validation Methods -------------------------------
\subsection{K-fold cross validation methods}
Randomly divides the dataset into k subsets. Them, for each subset :\\
\begin{enumerate}
    \item Learn a model on objects that are not in the subset
    \item Compute prediction on object that are in the subset
    \item Compute error.
\end{enumerate}
Them, report mean error over all the subsets.\\
\textcolor{red}{\textbf{Leave-one-out method}: N-fold cross validation where "N" is size of the dataset.}

  \begin{table}[!h]
    \begin{center}
    \begin{tabular}{| m{8em}| m{15em}|}
    \hline
    \rowcolor{vert.g} \textbf{Advantages}     &  \begin{itemize}
                                                        \item \textbf{Unbiased} (k=N)
                                                        \item \textbf{Low variance} (k<N)
                                                        \item \textbf{Fasten} (5 to 10 models to main) (k<N)
                                                 \end{itemize}\\ \hline 
    \rowcolor{red.g} \textbf{Drawbacks}       &  \begin{itemize}
                                                        \item \textbf{slow} (k = N (N models to train))
                                                        \item \textbf{High variance} (Highly dependent to dataset.
                                                        \item Potentially biased (k = 5,10)
                                                  \end{itemize}\\ \hline
    \end{tabular}
    \end{center}
    \end{table}

    
%-------------------------------------------- Boostrap Method -------------------------------
\subsection{Boostrap Method}
Randomly into the "k" subset in wich the dataset has been divided.\\
Boostrap error estimate:
\begin{enumerate}
    \item For i = 1 to k : 
    \begin{itemize}
        \item Learn a model on a boostrap sample Bi.
        \item compute the resubtitution error for each models
    \end{itemize}
    \item Compute the average over all model.
\end{enumerate}
Them, report mean error over all the subsets.\\
\textcolor{red}{\textbf{Leave-one-out method}: N-fold cross validation where "N" is size of the dataset.}

  \begin{table}[!h]
    \begin{center}
    \begin{tabular}{| m{8em}| m{15em}|}
    \hline
    \rowcolor{vert.g} \textbf{Advantages}     &  \begin{itemize}
                                                        \item \textbf{low variance}
                                                 \end{itemize}\\ \hline 
    \rowcolor{red.g} \textbf{Drawbacks}       &  \begin{itemize}
                                                        \item \textbf{Biais due to non distinst instances in a sample}
                                                        \item \textbf{Slow} 
                                                  \end{itemize}\\ \hline
    \end{tabular}
    \end{center}
    \end{table}
    
    \subsection{Model evaluation}
    \textbf{Large dataset} used test set method: randomly divide your data set in 3 points: LS VS and TS
      \begin{table}[!h]
    \begin{center}
    \begin{tabular}{| m{8em}| m{15em}|}
    \hline
    \rowcolor{vert.g} \textbf{Large dataset}     &  \begin{enumerate}
                                                        \item Train on LS your different models.
                                                        \item select the best based on its perfo on VS
                                                        \item retair on \textbf{ls} + \textbf{vS}
                                                        \item test it on \textbf{TS} : performance assesment
                                                        \item final model 
                                                 \end{enumerate}\\ \hline 
    \rowcolor{red.g} \textbf{small dataset}       &  \begin{enumerate}
                                                        \item \textbf{Divided the dataset} into k folds
                                                        \item \textbf{Learning} set  
                                                        \item train 
                                                        \item test performance
    
                                                  \end{enumerate}\\ \hline
                                                      \end{tabular}
    \end{center}
    \end{table}
    
      \begin{table}[!h]
    \begin{center}
    \begin{tabular}{| m{8em}| m{15em}|}
    \hline
     \rowcolor{vert.g} \textbf{small dataset}     &  \begin{enumerate}
                                                        \item divided dataset into 2 part
                                                        \item train and used k-fold cv on the first one to choose the best model
                                                        \item Test on test set to asses perfo
                                                 \end{enumerate}\\ \hline 
    \end{tabular}
    \end{center}
    \end{table}
    